\documentclass[]{article}

\include{../macros/main.tex}

\title{1111 Linux Operating System Project1}

\begin{document}
\maketitle
\begin{markdown}

# Hands-on

## Try Compile and Run Linux Kernel

### Compile Linux Kernel 6.0.5
``` ={shell}
PROJECT=/home/cliff/projects/course-linux-operation-system/project1
cd $PROJECT

# Install compile tools
sudo apt install build-essential fakeroot libncurses5-dev \
                 libssl-dev ccache flex bison libelf-dev bc

# Download
git submodule add https://github.com/torvalds/linux
cd $PROJECT/linux
git checkout 3829606fc5dffeccdf80aebeed3aa75255257f35

# Test compilation
make ARCH=x86_64 x86_64_defconfig
make -j$(nproc)
```

### Compile Busybox 1.35.0
``` ={shell}
cd $PROJECT

# Download
git submodule add https://github.com/mirror/busybox.git
cd $PROJECT/busybox

# Configuration
make defconfig

make menuconfig
# Busybox Settings → Build Options → [*] Build BusyBox as a static binary (no shared libs)

make -j$(nproc)
```

#### Generate initramfs
``` ={shell}
cd $PROJECT
mkdir initramfs
cd initramfs
mkdir -p bin sbin etc proc sys usr/bin usr/sbin
cp -av ../busybox/_install/* .
cat <<EOT > init
#!/bin/sh

mount -t proc none /proc
mount -t sysfs none /sys

echo -e "\nBoot took \$(cut -d' ' -f1 /proc/uptime) seconds\n"

mkdir -p /mnt/host_share
mount -t 9p -o trans=virtio host_share /mnt/host_share -oversion=9p2000.L

exec /bin/sh -c '/mnt/host_share/main; exec sh'
EOT
chmod +x init
find . -print0 | cpio --null -ov --format=newc | gzip -9 > ../initramfs.cpio.gz
```

### Run
``` ={shell}
LINUX_KERNEL_PATH=$PROJECT/linux/arch/x86_64/boot/bzImage
INITRAMFS_PATH=$PROJECT/initramfs.cpio.gz
qemu-system-x86_64 \
  -kernel $LINUX_KERNEL_PATH \
  -initrd $INITRAMFS_PATH \
  -nographic \
  -enable-kvm \
  -append "console=ttyS0" \
  -virtfs local,path=$PROJECT,security_model=passthrough,mount_tag=host_share
```

## Task1, 2: Add `sys_segment_info` System Call

### Modify Kernel: add system call
``` ={shell}
cat <<EOT > $PROJECT/sys_segment_info.c
#include <linux/types.h>
#include <linux/syscalls.h>
#include <linux/kernel.h>
#include <linux/sched.h>
#include <linux/ptrace.h>
#include <linux/thread_info.h>
#include <asm/current.h>

#include <linux/segment_info.h>

SYSCALL_DEFINE1(segment_info, struct segment_info *, dsi)
{
  struct vm_area_struct *vma = current->mm->mmap;
  struct segment_info si = {
      .ma_size = 0,
      .start_code = current->mm->start_code,
      .end_code = current->mm->end_code,
      .start_data = current->mm->start_data,
      .end_data = current->mm->end_data,
      .start_brk = current->mm->start_brk,
      .brk = current->mm->brk,
      .mmap_base = current->mm->mmap_base,
      .thread_sp = current_user_stack_pointer(),
  };

  while (vma)
  {
    struct ma_struct *ma = si.ma + si.ma_size++;
    *ma = (struct ma_struct){
        .vm_start = vma->vm_start,
        .vm_end = vma->vm_end,
        .name = "\0",
    };

    if (vma->vm_file)
      // fs/d_path.c
      d_path(&vma->vm_file->f_path, ma->name, sizeof(ma->name));
    else if (vma->vm_ops && vma->vm_ops->name)
      strcpy(ma->name, vma->vm_ops->name(vma));

    vma = vma->vm_next;
  }

  if (copy_to_user(dsi, &si, sizeof(si)))
    return -1;

  return current->pid;
}
EOT

cat <<EOT > $PROJECT/segment_info.h
struct ma_struct
{
	unsigned long vm_start, vm_end;
	char name[24];
};

struct segment_info
{
	unsigned long start_code, end_code;
	unsigned long start_data, end_data;
	unsigned long start_brk, brk;
	unsigned long start_stack, end_stack;
	unsigned long thread_sp;
	unsigned long mmap_base;
	unsigned long ma_size;
	struct ma_struct ma[24];
};
EOT
```

### Add source file to options and compile
``` ={shell}
ln -s $PROJECT/sys_segment_info.c \
      $PROJECT/linux/arch/x86/kernel/sys_segment_info.c
ln -s $PROJECT/segment_info.h \
      $PROJECT/linux/include/linux/segment_info.h
echo "451	common	segment_info		sys_segment_info" \
      >> $PROJECT/linux/arch/x86/entry/syscalls/syscall_64.tbl
echo "obj-y                   += sys_segment_info.o" \
      >> $PROJECT/linux/arch/x86/kernel/Makefile
cd $PROJECT/linux
make -j$(nproc)
```

% echo "struct segment_info;\nasmlinkage long sys_segment_info(pid_t tid, struct segment_info* si);" >> $PROJECT/linux/include/linux/syscalls.h

### Modify Guest User Space Code: add multi-thread
``` ={c++}
cat <<EOT > $PROJECT/main.c
#include <stdio.h>
#include <stdlib.h>
#include <string.h>

#include <sys/mman.h>
#include <sys/types.h>
#include <sys/syscall.h>
#include <pthread.h>

#include "segment_info.h"

char bss[4];                    // bss
char data[5] = "data";          // data
char *heap;                     // heap
char *mmmap;                    // mmap
char *code() { return "code"; } // code

char intersect(unsigned long a1, unsigned long a2, unsigned long b1, unsigned long b2)
{
  return (b2 > a1 && b2 <= a2) || (b1 >= a1 && b1 < a2) || (b1 <= a1 && b2 > a2);
}

void print_segment_info(char *thread_name)
{
  setvbuf(stdout, NULL, _IOFBF, 16384);
  char stack[6] = "stack"; // stack

  struct segment_info si;
  printf("˅˅˅˅˅˅˅˅˅˅˅˅˅˅˅ [User Mode] %s thread id: %d ˅˅˅˅˅˅˅˅˅˅˅˅˅˅\n", thread_name, syscall(451, &si));
  printf(">>>>>> [%s]:\t\t'%p'\n", code(), &code);
  printf(">>>>>> [%s]:\t\t'%p'\n", data, &data);
  printf(">>>>>> [%s]:\t\t'%p'\n", bss, &bss);
  printf(">>>>>> [%s]:\t\t'%p'\n", heap, heap);
  printf(">>>>>> [%s]:\t\t'%p'\n", mmmap, mmmap);
  printf(">>>>>> [%s]:\t\t'%p'\n", stack, &stack);
  printf(">>>>>>>>>>>>>>>>>>>>>>>>>>>>><<<<<<<<<<<<<<<<<<<<<<<<<<<<<\n");
  printf(">>>>>> <start_code>:\t'%p'\n", si.start_code);
  printf(">>>>>> <end_code>:\t'%p'\n", si.end_code);
  printf(">>>>>> <start_data>:\t'%p'\n", si.start_data);
  printf(">>>>>> <end_data>:\t'%p'\n", si.end_data);
  printf(">>>>>> <start_brk>:\t'%p'\n", si.start_brk);
  printf(">>>>>> <brk>:\t\t'%p'\n", si.brk);
  printf(">>>>>> <mmap_base>:\t'%p'\n", si.mmap_base);
  printf(">>>>>> <thread_sp>:\t'%p'\n", si.thread_sp);
  printf(">>>>>>>>>>>>>>>>>>>>>>>>>>>>><<<<<<<<<<<<<<<<<<<<<<<<<<<<<\n");

  for (int w = 0; w < si.ma_size; ++w)
  {
    char msg[100] = "unknown ";
    if (intersect(si.ma[w].vm_start, si.ma[w].vm_end, si.start_code, si.end_code))
      strcpy(msg, "code/text segment ");
    else if (intersect(si.ma[w].vm_start, si.ma[w].vm_end, si.start_data, si.end_data))
      strcpy(msg, "data segment ");
    else if (intersect(si.ma[w].vm_start, si.ma[w].vm_end, si.end_data, si.start_brk))
      strcpy(msg, "bss segment ");
    else if (intersect(si.ma[w].vm_start, si.ma[w].vm_end, si.start_brk, si.brk))
      strcpy(msg, "heap segment ");
    else if (intersect(si.ma[w].vm_start, si.ma[w].vm_end, si.thread_sp, si.thread_sp))
      sprintf(msg, "%s stack segment ", thread_name);
    else if (intersect(si.ma[w].vm_start, si.ma[w].vm_end, si.brk, si.mmap_base))
      strcpy(msg, "mmap segment(shared library, thread stack...) ");

    if (si.ma[w].name)
      sprintf(msg, "%s%s", msg, si.ma[w].name);

    printf(">>>>>> '%p'-'%p' %s\n", si.ma[w].vm_start, si.ma[w].vm_end, msg);
  }

  printf("^^^^^^^^^^^^^^^^^^^^^^^^^^^^^^^^^^^^^^^^^^^^^^^^^^^^^^^^^^\n");
  fflush(stdout);
}

int main()
{
  heap = (char *)malloc(sizeof(char) * 5);
  mmmap = mmap(NULL, 5 * sizeof(char), PROT_READ | PROT_WRITE, MAP_PRIVATE | MAP_ANONYMOUS, 0, 0);
  strcpy(bss, "bss");
  strcpy(heap, "heap");
  strcpy(mmmap, "mmap");

  print_segment_info("main");

  pthread_t t1, t2;
  pthread_create(&t1, NULL, print_segment_info, "t1");
  pthread_create(&t2, NULL, print_segment_info, "t2");
  pthread_join(t1, NULL);
  pthread_join(t2, NULL);
  free(heap);
}
EOT
```

### Compile and Run Virtual Machine
``` ={shell}
cd $PROJECT
gcc -Wno-format -Wno-incompatible-pointer-types \
    -Wno-implicit-function-declaration -Wno-error=unused-result \
    -o main main.c -static -g

LINUX_KERNEL_PATH=$PROJECT/linux/arch/x86_64/boot/bzImage
INITRAMFS_PATH=$PROJECT/initramfs.cpio.gz
qemu-system-x86_64 \
  -kernel $LINUX_KERNEL_PATH \
  -initrd $INITRAMFS_PATH \
  -nographic \
  -enable-kvm \
  -append "console=ttyS0" \
  -virtfs local,path=$PROJECT,security_model=passthrough,mount_tag=host_share
```

#### Output

``` ={shell}
˅˅˅˅˅˅˅˅˅˅˅˅˅˅˅ [User Mode] main thread id: 77 ˅˅˅˅˅˅˅˅˅˅˅˅˅˅
>>>>>> [code]:          '0x401845'
>>>>>> [data]:          '0x4e1110'
>>>>>> [bss]:           '0x4e33d0'
>>>>>> [heap]:          '0x1d16770'
>>>>>> [mmap]:          '0x7fa3955d4000'
>>>>>> [stack]:         '0x7ffc10a9eb5a'
>>>>>>>>>>>>>>>>>>>>>>>>>>>>><<<<<<<<<<<<<<<<<<<<<<<<<<<<<
>>>>>> <start_code>:    '0x401000'
>>>>>> <end_code>:      '0x4b099d'
>>>>>> <start_data>:    '0x4dd768'
>>>>>> <end_data>:      '0x4e3370'
>>>>>> <start_brk>:     '0x1d15000'
>>>>>> <brk>:           '0x1d37000'
>>>>>> <mmap_base>:     '0x7fa3955d5000'
>>>>>> <thread_sp>:     '0x7ffc10a9e718'
>>>>>>>>>>>>>>>>>>>>>>>>>>>>><<<<<<<<<<<<<<<<<<<<<<<<<<<<<
>>>>>> '0x400000'-'0x401000' unknown
>>>>>> '0x401000'-'0x4b1000' code/text segment
>>>>>> '0x4b1000'-'0x4dd000' unknown
>>>>>> '0x4dd000'-'0x4e1000' data segment
>>>>>> '0x4e1000'-'0x4e4000' data segment
>>>>>> '0x4e4000'-'0x4ea000' bss segment
>>>>>> '0x1d15000'-'0x1d37000' heap segment
>>>>>> '0x7fa3955d4000'-'0x7fa3955d5000' mmap segment(shared library, thread stack...)
>>>>>> '0x7ffc10a7f000'-'0x7ffc10aa0000' main stack segment
>>>>>> '0x7ffc10bc1000'-'0x7ffc10bc5000' unknown [vvar]
>>>>>> '0x7ffc10bc5000'-'0x7ffc10bc7000' unknown [vdso]
^^^^^^^^^^^^^^^^^^^^^^^^^^^^^^^^^^^^^^^^^^^^^^^^^^^^^^^^^^
˅˅˅˅˅˅˅˅˅˅˅˅˅˅˅ [User Mode] t2 thread id: 79 ˅˅˅˅˅˅˅˅˅˅˅˅˅˅
>>>>>> [code]:          '0x401845'
>>>>>> [data]:          '0x4e1110'
>>>>>> [bss]:           '0x4e33d0'
>>>>>> [heap]:          '0x1d16770'
>>>>>> [mmap]:          '0x7fa3955d4000'
>>>>>> [stack]:         '0x7fa394dd215a'
>>>>>>>>>>>>>>>>>>>>>>>>>>>>><<<<<<<<<<<<<<<<<<<<<<<<<<<<<
>>>>>> <start_code>:    '0x401000'
>>>>>> <end_code>:      '0x4b099d'
>>>>>> <start_data>:    '0x4dd768'
>>>>>> <end_data>:      '0x4e3370'
>>>>>> <start_brk>:     '0x1d15000'
>>>>>> <brk>:           '0x1d37000'
>>>>>> <mmap_base>:     '0x7fa3955d5000'
>>>>>> <thread_sp>:     '0x7fa394dd1d18'
>>>>>>>>>>>>>>>>>>>>>>>>>>>>><<<<<<<<<<<<<<<<<<<<<<<<<<<<<
>>>>>> '0x400000'-'0x401000' unknown
>>>>>> '0x401000'-'0x4b1000' code/text segment
>>>>>> '0x4b1000'-'0x4dd000' unknown
>>>>>> '0x4dd000'-'0x4e1000' data segment
>>>>>> '0x4e1000'-'0x4e4000' data segment
>>>>>> '0x4e4000'-'0x4ea000' bss segment
>>>>>> '0x1d15000'-'0x1d37000' heap segment
>>>>>> '0x7fa3945d2000'-'0x7fa3945d3000' mmap segment(shared library, thread stack...)
>>>>>> '0x7fa3945d3000'-'0x7fa394dd3000' t2 stack segment
>>>>>> '0x7fa394dd3000'-'0x7fa394dd4000' mmap segment(shared library, thread stack...)
>>>>>> '0x7fa394dd4000'-'0x7fa3955d5000' mmap segment(shared library, thread stack...)
>>>>>> '0x7ffc10a7f000'-'0x7ffc10aa0000' unknown
>>>>>> '0x7ffc10bc1000'-'0x7ffc10bc5000' unknown [vvar]
>>>>>> '0x7ffc10bc5000'-'0x7ffc10bc7000' unknown [vdso]
^^^^^^^^^^^^^^^^^^^^^^^^^^^^^^^^^^^^^^^^^^^^^^^^^^^^^^^^^^
˅˅˅˅˅˅˅˅˅˅˅˅˅˅˅ [User Mode] t1 thread id: 78 ˅˅˅˅˅˅˅˅˅˅˅˅˅˅
>>>>>> [code]:          '0x401845'
>>>>>> [data]:          '0x4e1110'
>>>>>> [bss]:           '0x4e33d0'
>>>>>> [heap]:          '0x1d16770'
>>>>>> [mmap]:          '0x7fa3955d4000'
>>>>>> [stack]:         '0x7fa3955d315a'
>>>>>>>>>>>>>>>>>>>>>>>>>>>>><<<<<<<<<<<<<<<<<<<<<<<<<<<<<
>>>>>> <start_code>:    '0x401000'
>>>>>> <end_code>:      '0x4b099d'
>>>>>> <start_data>:    '0x4dd768'
>>>>>> <end_data>:      '0x4e3370'
>>>>>> <start_brk>:     '0x1d15000'
>>>>>> <brk>:           '0x1d37000'
>>>>>> <mmap_base>:     '0x7fa3955d5000'
>>>>>> <thread_sp>:     '0x7fa3955d2d18'
>>>>>>>>>>>>>>>>>>>>>>>>>>>>><<<<<<<<<<<<<<<<<<<<<<<<<<<<<
>>>>>> '0x400000'-'0x401000' unknown
>>>>>> '0x401000'-'0x4b1000' code/text segment
>>>>>> '0x4b1000'-'0x4dd000' unknown
>>>>>> '0x4dd000'-'0x4e1000' data segment
>>>>>> '0x4e1000'-'0x4e4000' data segment
>>>>>> '0x4e4000'-'0x4ea000' bss segment
>>>>>> '0x1d15000'-'0x1d37000' heap segment
>>>>>> '0x7fa3945d2000'-'0x7fa3945d3000' mmap segment(shared library, thread stack...)
>>>>>> '0x7fa3945d3000'-'0x7fa394dd3000' mmap segment(shared library, thread stack...)
>>>>>> '0x7fa394dd3000'-'0x7fa394dd4000' mmap segment(shared library, thread stack...)
>>>>>> '0x7fa394dd4000'-'0x7fa3955d5000' t1 stack segment
>>>>>> '0x7ffc10a7f000'-'0x7ffc10aa0000' unknown
>>>>>> '0x7ffc10bc1000'-'0x7ffc10bc5000' unknown [vvar]
>>>>>> '0x7ffc10bc5000'-'0x7ffc10bc7000' unknown [vdso]
^^^^^^^^^^^^^^^^^^^^^^^^^^^^^^^^^^^^^^^^^^^^^^^^^^^^^^^^^^
```

# Prior Knowledge

## Task, Thread, Process in Linux

在 Linux 作業系統中,執行的最小單位稱為 Task,資料結構是由 include/linux/sched.h#L727 (v6.0.5) 下的 `task_struct` 定義,可以看作是一個 process descriptor。

## SYSCALL_DEFINEx

## Page Table

## VMA

## initrd, initramfs

## copy_to_user, copy_from_user

## Linux Copy On Write Memory

## task_struct
### mm_struct

## thread_info

# Reference

## Build Linux Kernel

- [Building a Custom Linux Kernel & Debugging via QEMU + GDB](https://www.josehu.com/memo/2021/01/02/linux-kernel-build-debug.html)
- [Prepare the environment for developing Linux kernel with qemu.](https://medium.com/@daeseok.youn/prepare-the-environment-for-developing-linux-kernel-with-qemu-c55e37ba8ade)
- [How to Build A Custom Linux Kernel For Qemu](https://mudongliang.github.io/2017/09/12/how-to-build-a-custom-linux-kernel-for-qemu.html)
- [Build the Linux kernel and Busybox and run them on QEMU](https://www.zachpfeffer.com/single-post/build-the-linux-kernel-and-busybox-and-run-on-qemu)

## Build BusyBox with `host_share` storage
- [How to qemu-arm with busybox linux and shared folder](https://gist.github.com/franzflasch/132139b8ba798066394e62b3f7532f7a)

## Add System Call
- [Adding a New System Call](https://www.kernel.org/doc/html/v6.0/process/adding-syscalls.html)
- [How to pass parameters to Linux system call?](https://stackoverflow.com/questions/53735886/how-to-pass-parameters-to-linux-system-call)
- [System Call (系統呼叫)](https://hackmd.io/@combo-tw/BJPoAcqQS)
- [System calls in the Linux kernel. Part 1.](https://0xax.gitbooks.io/linux-insides/content/SysCall/linux-syscall-1.html)

## Misc
- [Which Linux syscall is used to get a thread's ID?](https://stackoverflow.com/questions/19350212/which-linux-syscall-is-used-to-get-a-threads-id)
- [How can we get the starting address of task_struct of a process](https://unix.stackexchange.com/questions/568314/how-can-we-get-the-starting-address-of-task-struct-of-a-process)

### To Be Organized

#### Tier 1
- [Address Space](https://linux-kernel-labs.github.io/refs/heads/master/lectures/address-space.html#)
- [Chapter 3  Page Table Management](https://www.kernel.org/doc/gorman/html/understand/understand006.html#fig)
- [How The Kernel Manages Your Memory](https://manybutfinite.com/post/how-the-kernel-manages-your-memory/)
- [Page Tables](https://github.com/lorenzo-stoakes/linux-vm-notes/blob/master/sections/page-tables.md)
- [Process Address Space](https://github.com/lorenzo-stoakes/linux-vm-notes/blob/master/sections/process.md)
- [Virtual Memory (虛擬記憶體)](https://hackmd.io/@combo-tw/BJlTwJUABB)
- [OS Process & Thread (user/kernel) 筆記](https://medium.com/@yovan/os-process-thread-user-kernel-筆記-aa6e04d35002)
- [Linux 核心 Copy On Write - Memory Region](https://hackmd.io/@linD026/Linux-kernel-COW-memory-region#reverse-mapping-anon_vma)
- [Linux 核心 Copy On Write 實作機制](https://hackmd.io/@linD026/Linux-kernel-COW-Copy-on-Write#Virtual-Address-to-Phyiscal-Address-and-back)
- [Linux 核心設計: Memory](https://hackmd.io/@RinHizakura/rJTl9K5tv)
- [Linux 核心設計: 記憶體管理](https://hackmd.io/@sysprog/linux-memory)
- [Linux作業系統學習筆記(三)核心初始化](https://ty-chen.github.io/linux-kernel-zero-process/)
- [Linux的程序地址空間[三]](https://zhuanlan.zhihu.com/p/68398179)
- [Linux的程序地址空間[二] - VMA](https://zhuanlan.zhihu.com/p/67936075)
- [Linux程序描述符task_struct結構體詳解--Linux程序的管理與排程(一)](https://blog.csdn.net/gatieme/article/details/51383272)
- [Linux程序核心棧與thread_info結構詳解--Linux程序的管理與排程(九)](https://blog.csdn.net/gatieme/article/details/51577479)
- [Linux程序棧空間大小](https://www.tiehichi.site/2020/10/22/Linux进程栈空间大小/)
- [Linux程序棧空間大小](https://zhuanlan.zhihu.com/p/530357476)
- [Linux記憶體管理第三章 -- 頁表管理(Page Table Management)](https://blog.csdn.net/weixin_38537730/article/details/104252794)
- [分享一個關於pthread執行緒棧在mm_struct裡面的分佈問題](http://bbs.chinaunix.net/forum.php?mod=viewthread&tid=2018590)

#### Tier 2
- ["current" in Linux kernel code](https://stackoverflow.com/questions/22346545/current-in-linux-kernel-code)
- [How to get the physical address from the logical one in a Linux kernel module?](https://stackoverflow.com/questions/6252063/how-to-get-the-physical-address-from-the-logical-one-in-a-linux-kernel-module)
- [Is stack memory contiguous?](https://stackoverflow.com/questions/5086577/is-stack-memory-contiguous)
- [Is stack memory contiguous physically in Linux?](https://stackoverflow.com/questions/49595006/is-stack-memory-contiguous-physically-in-linux)
- [Linux Kernel — get page global directory and analyze the result](https://medium.com/@lsc830621/linux-kernel-get-page-global-directory-and-analyze-the-result-68a015b5693c)
- [NCTU OSDI Dicussion - Memory Management III](https://hackmd.io/@JingWang/BJOhnNMcL)
- [Where are the stacks for the other threads located in a process virtual address space?](https://stackoverflow.com/questions/44858528/where-are-the-stacks-for-the-other-threads-located-in-a-process-virtual-address)
- [(三)程序各種id:pid、pgid、sid、全域性pid、區域性pid](https://blog.csdn.net/qq_33160790/article/details/81346663)
- [【原創】(十三)Linux記憶體管理之vma/malloc/mmap](https://www.cnblogs.com/LoyenWang/p/12037658.html)
- [/proc/<pid>/maps簡要分析](https://www.cnblogs.com/arnoldlu/p/10272466.html)
- [linux 記憶體管理(8) —記憶體描述符(mm_struct)](https://blog.csdn.net/weixin_41028621/article/details/104455327)
- [Linux程序地址管理之mm_struct](https://www.cnblogs.com/rofael/archive/2013/04/13/3019153.html)
- [mm_struct簡介](https://blog.csdn.net/LF_2016/article/details/54346121)

\end{markdown}
\end{document}